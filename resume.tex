\documentclass[10pt, letterpaper]{article}

% Packages:
\usepackage[
    ignoreheadfoot, % set margins without considering header and footer
    top=1.5 cm, % seperation between body and page edge from the top
    bottom=1.5 cm, % seperation between body and page edge from the bottom
    left=1.5 cm, % seperation between body and page edge from the left
    right=1.5 cm, % seperation between body and page edge from the right
    footskip=0.5 cm, % seperation between body and footer
    % showframe % for debugging 
]{geometry} % for adjusting page geometry
\usepackage[explicit]{titlesec} % for customizing section titles
\usepackage{tabularx} % for making tables with fixed width columns
\usepackage{array} % tabularx requires this
\usepackage[dvipsnames]{xcolor} % for coloring text
\definecolor{primaryColor}{RGB}{0, 79, 144} % define primary color
\usepackage{enumitem} % for customizing lists
\usepackage{fontawesome5} % for using icons
\usepackage{amsmath} % for math
\usepackage[
    pdftitle={Amit Matth's CV},
    pdfauthor={Amit Matth},
    pdfcreator={LaTeX with RenderCV},
    colorlinks=true,
    urlcolor=primaryColor
]{hyperref} % for links, metadata and bookmarks
\usepackage[pscoord]{eso-pic} % for floating text on the page
\usepackage{calc} % for calculating lengths
\usepackage{bookmark} % for bookmarks
\usepackage{lastpage} % for getting the total number of pages
\usepackage{changepage} % for one column entries (adjustwidth environment)
\usepackage{paracol} % for two and three column entries
\usepackage{ifthen} % for conditional statements
\usepackage{needspace} % for avoiding page brake right after the section title
\usepackage{iftex} % check if engine is pdflatex, xetex or luatex

% Ensure that generate pdf is machine readable/ATS parsable:
\ifPDFTeX
    \input{glyphtounicode}
    \pdfgentounicode=1
    \usepackage[T1]{fontenc}
    \usepackage[utf8]{inputenc}
    \usepackage{lmodern}
\fi

\usepackage[default, type1]{sourcesanspro} 

% Some settings:
\AtBeginEnvironment{adjustwidth}{\partopsep0pt} % remove space before adjustwidth environment
\pagestyle{empty} % no header or footer
\setcounter{secnumdepth}{0} % no section numbering
\setlength{\parindent}{0pt} % no indentation
\setlength{\topskip}{0pt} % no top skip
\setlength{\columnsep}{0.15cm} % set column seperation
\makeatletter

\titleformat{\section}{
    % avoid page braking right after the section title
    \needspace{4\baselineskip}
    % make the font size of the section title large and color it with the primary color
    \Large\color{primaryColor}
}{
}{
}{
    % print bold title, give 0.15 cm space and draw a line of 0.8 pt thickness
    % from the end of the title to the end of the body
    \textbf{#1}\hspace{0.15cm}\titlerule[0.8pt]\hspace{-0.1cm}
}[] % section title formatting

\titlespacing{\section}{
    % left space:
    -1pt
}{
    % top space:
    0.3 cm
}{
    % bottom space:
    0.3 cm
} % section title spacing

% \renewcommand\labelitemi{$\vcenter{\hbox{\small$\bullet$}}$} % custom bullet points
\newenvironment{highlights}{
    \begin{itemize}[
        topsep=0.10 cm,
        parsep=0.15 cm,
        partopsep=0pt,
        itemsep=0pt,
        leftmargin=0.4 cm + 10pt
    ]
}{
    \end{itemize}
} % new environment for highlights

\newenvironment{highlightsforbulletentries}{
    \begin{itemize}[
        topsep=0.1 cm,
        parsep=0.10 cm,
        partopsep=0pt,
        itemsep=0pt,
        leftmargin=10pt
    ]
}{
    \end{itemize}
} % new environment for highlights for bullet entries


\newenvironment{onecolentry}{
    \begin{adjustwidth}{
        0.2 cm + 0.00001 cm
    }{
        0.2 cm + 0.00001 cm
    }
}{
    \end{adjustwidth}
} % new environment for one column entries

\newenvironment{twocolentry}[2][]{
    \onecolentry
    \def\secondColumn{#2}
    \setcolumnwidth{\fill, 2.5 cm}
    \begin{paracol}{2}
}{
    \switchcolumn \raggedleft \secondColumn
    \end{paracol}
    \endonecolentry
} % new environment for two column entries


\newenvironment{header}{
    \setlength{\topsep}{0pt}\par\kern\topsep\centering\color{primaryColor}\linespread{1.5}
}{
    \par\kern\topsep
} % new environment for the header


% save the original href command in a new command:
\let\hrefWithoutArrow\href

% new command for external links:
\renewcommand{\href}[2]{\hrefWithoutArrow{#1}{\ifthenelse{\equal{#2}{}}{ }{#2 }\raisebox{.15ex}{\footnotesize \faExternalLink*}}}


\begin{document}

    \newcommand{\AND}{\unskip
        \cleaders\copy\ANDbox\hskip\wd\ANDbox
        \ignorespaces
    }
    \newsavebox\ANDbox
    \sbox\ANDbox{}

\begin{header}
    \noindent
    \begin{minipage}[t]{0.55\textwidth} % Left side
        \fontsize{25 pt}{25 pt}\selectfont
        \textbf{Amit Matth}

        \vspace{0.05 cm}

        \normalsize
        \textit{Application Developer \& System Engineer}
    \end{minipage}%
    \hfill
    \begin{minipage}[t]{0.42\textwidth} % Right side
        \raggedleft
        \footnotesize
        {\faMapMarker*} Bhiwani, Haryana, India \\
        \hrefWithoutArrow{mailto:amitmatth121@gmail.com}{{\faEnvelope[regular]} amitmatth121@gmail.com} \, | \,
        {\faPhone*} 9813662025 \\
        \hrefWithoutArrow{https://www.linkedin.com/in/amit-kumar101/}{{\faLinkedinIn} amit-matth} \, | \,
        \hrefWithoutArrow{https://github.com/Amit-Matth}{{\faGithub} Amit-Matth}
    \end{minipage}
\end{header}

    \vspace{0.05 cm}

    \small
%     \section{Skills}
        
%        \begin{highlights}
%     \item \textbf{Programming Languages:} Kotlin, Dart, Java, XML
% \end{highlights}

% \vspace{0.01 cm}

% \begin{highlights}
%     \item \textbf{Databases:} SQLite, Room Database, Preferences Datastore, Hive Database
% \end{highlights}

% \vspace{0.01 cm}

% \begin{highlights}
%     \item \textbf{Cloud \& APIs:} Firebase (Authentication, Realtime Database), Google Maps API, Gemini API
% \end{highlights}

%         \vspace{0.01 cm}

%          \begin{highlights}
%             \item \textbf{Tools \& Technologies:} Flutter, Jetpack Compose, Material Design, Android Studio, Android SDK, MVVM, Volley, Glide
%         \end{highlights}

%         \vspace{0.01 cm}

%          \begin{highlights}
%             \item \textbf{Version Control:} Git \& Github
%         \end{highlights}

%         \vspace{0.01 cm}

%          \begin{highlights}
%            \item  \textbf{Testing:} JUnit, Mockito
%         \end{highlights}

%      \vspace{0.01 cm}


\section{Skills}

\begin{highlights}
    \item \textbf{Programming Languages:} C, C++, Python
\end{highlights}

\vspace{0.01 cm}

\begin{highlights}
    \item \textbf{Embedded \& Systems:} RISC-V, Embedded Systems
\end{highlights}

\vspace{0.01 cm}

\begin{highlights}
    \item \textbf{Mobile Development:} Kotlin, Java, Flutter, Jetpack Compose
\end{highlights}

\vspace{0.01 cm}

\begin{highlights}
    \item \textbf{Databases:} SQLite, Room, Firebase Realtime Database, Preferences Datastore
\end{highlights}

\vspace{0.01 cm}

\begin{highlights}
    \item \textbf{Tools \& Toolchains:} Git, GitHub, GDB, Spike, QEMU, Verilator, Make, riscv64-unknown-elf
\end{highlights}

\vspace{0.01 cm}

\begin{highlights}
    \item \textbf{Operating Systems:} Linux (Ubuntu, WSL)
\end{highlights}
\vspace{0.01 cm}
\begin{highlights}
    \item \textbf{Testing \& Validation:} JUnit, Mockito, Bare-metal testing, Spike \& QEMU simulation
\end{highlights}



\section{Relevant Experience}

\begin{onecolentry}
    \textbf{Google Summer of Code (GSoC) 2025 – International Catrobat Association(Paintroid-Flutter)} 
    \hfill 
    \href{https://github.com/Catrobat/Paintroid-Flutter}{GitHub} \, | \,
    \href{https://gist.github.com/Amit-Matth/bf5981e2bd2237161c0e16f3e3e3959a#file-gsoc-2025-md}{My Contributions}

    \begin{highlights}
        \item Added Text Tool, Shapes, Color Picker, and other drawing tools.
        \item Wrote tests to make sure the app works correctly and is easy to maintain.
        \item \textbf{Technologies Used:} Flutter, Dart, Kotlin
    \end{highlights}
\end{onecolentry}

\vspace{0.01 cm}

\section{Projects}

\begin{onecolentry}
    \textbf{RISC-V DSP / Neural Network Library} \, | \, C, RISC-V, DSP, ML
    \hfill \href{https://github.com/Amit-Matth/riscv-dsp-lib}{GitHub}

    \begin{highlights}
        \item Built a library to run audio processing and basic AI on RISC-V chips.
        \item Created fast and efficient code for mathematical calculations.
        \item Implemented a system to run simple machine learning models.
        \item Built a system to process audio in real-time.
        \item Made sure the project runs on different systems like Simulators and Native hardware.
        \item \textbf{Technologies:} C (C99), RISC-V, DSP, ML, Spike, QEMU, Make, PulseAudio
    \end{highlights}
\end{onecolentry}

\vspace{0.1 cm}

\begin{onecolentry}
    \textbf{RISC-V Compliance Automation Suite} \, | \, Python, RISC-V, Automation
    \hfill \href{https://github.com/Amit-Matth/riscv_compliance_automation}{GitHub}

    \begin{highlights}
        \item Created a Python tool to automatically check if RISC-V processors are working correctly.
        \item Automated the setup and configuration process for different types of chips.
        \item Built a system to run tests automatically and find errors, which saves a lot of manual work.
        \item \textbf{Technologies:} Python, RISC-V, Scripting, Git/GitHub
    \end{highlights}
\end{onecolentry}


\vspace{0.1 cm}

\begin{onecolentry}
    \textbf{Challenge Monitor – Android App} \, | \, Kotlin, XML, MVVM, SQLite, SharedPreferences, WorkManager, AlarmManager, Glide
    \hfill \href{https://github.com/Amit-Matth/Challenge-Monitor}{GitHub}

    \begin{highlights}
        \item Created an Android app to help users track personal habits and challenges.
        \item Added features to log daily progress and view it on a calendar. 
        \item Included options to skip days and track streaks even without internet.
    \end{highlights}
\end{onecolentry}

\vspace{0.1 cm}

\begin{onecolentry}
    \textbf{Portfolio – Flutter Web App} \, | \, Flutter, Firebase
    \hfill \href{https://github.com/Amit-Matth/Portfolio}{GitHub}

    \begin{highlights}
        \item Built a personal website to show my projects and skills.
        \item Integrated Firebase for contact form submissions.
    \end{highlights}
\end{onecolentry}

\section{Education}

\begin{onecolentry}
    \textbf{SD Senior Secondary School – Non-Medical (PCM)} 
    \hfill April 2022 - March 2023
    \begin{highlights}
        \item Scored \textbf{75\%} in Grade \textbf{12th} (HBSE)
    \end{highlights}
\end{onecolentry}

\vspace{0.01 cm}

\begin{onecolentry}
    \textbf{Janta High School} 
    \hfill April 2020 – March 2021
    \begin{highlights}
        \item Scored \textbf{100\%} in Grade \textbf{10th} (HBSE)
    \end{highlights}
\end{onecolentry}

\vspace{0.01 cm}

\end{document} 
